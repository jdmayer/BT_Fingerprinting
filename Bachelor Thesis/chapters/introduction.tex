\chapter{Introduction}
\label{cha:Introduction}

\section{Motivation}
In March 2018, the private data from tens of millions of users were leaked by the social media plattform Facebook.(\textcite{fbScandal}) This was not the first big leak of user information (e.g. Google Drive in 2014 (\textcite{googleDrive})), and it certainly will not be the last.\\\\
What is not widely known by less technically interested people, is that you do not need to have an account to leak personal information. Nowadays there are various ways of tracking internet users and collection information. From cookies to fingerprinting, there are always new innovations.\\\\
By now tracking became part of the everyday life online. Even advertisments use customizations as this strategy promises to lift sales by 10\% (\textcite{ariker15}). \\\\
There are few methods to check if you are being tracked and even fewer to avoid being so. Trying to avoid malicious websites, blocking third party cookies, using the incognito mode or not using specific services. But afterall none of these choices seems sufficient to grant real privacy. Therefore the best method seems to be, not using the internet at all.




%methoden, ergebnisse, ihre bedeutsamkeit - keine Inhaltsangabe
\section{Objectives}
\subsection{Analysis of different web browser fingerprinting techniques}
One of the objectives of this bachelor thesis is to take a closer look as the different web browser fingerprinting techniques, which are currently in use. 
First, some of the configurations and plugins, which are used most, are introduced to give a better explanation, why the browser leaks information and what it is necessary for. Further, the different methods of fingerprinting are explained and based on them the different techniques.
Based on the foundation set in these subchapters the techniques will be shown based on an application scenario. \\ 
\\
The most important points which are discussed regarding this objective are:
\begin{itemize}
	\item used configurations and plug-ins
	\item fingerprinting methods
	\item fingerprinting techniques
\end{itemize}

\subsection{Prototype}
Due to the outcome of the analysis of the different web browser fingerprinting techniques the best technique for the prototype will be chosen. The objective of the prototype is to show how web browser fingerprinting is implemented and how it works. Based on different configurations and plug-ins the prototype will create a hash which will help to re-identify users.\\\\
The prototype will be realized in form of a website which will include a fingerprinting script. It's functionality will cover following points:
\begin{itemize}
	\item reading configurations and plug-ins
	\item creating a hash
	\item (re-)identifying users
\end{itemize}

\section{Overview}
Following chapters discuss the stated content:\\\\
\hyperref[cha:foundation]{Chapter 2} discusses the basics of web browser fingerprinting, amongst others the different methods of fingerprinting. In order to compare them later on to analyse and eventually choose one of the methods to implement the prototype for chapter 4.\\
\hyperref[cha:requirements]{Chapter 3} will specify the requirements which are needed to outline a proper prototype.\\
\hyperref[cha:implementation]{Chapter 4} explains the implementation and design of the prototype.\\
\hyperref[cha:evaluation]{Chapter 5} concludes the previous chapter based on test cases and a proper evaluation.\\
\hyperref[cha:summary]{Chapter 6} summarizes the results of this bachelor thesis and states possible prospects.




