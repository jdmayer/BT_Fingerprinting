\chapter{Introduction}
\label{cha:Introduction}

\section{Motivation}

The internet plays a big role in our everyday life. It is used to search for information, stay in contact with other users and even to shop for items. Most of the users do not waste another thought on who could see what they are doing. Nobody has time to read the endless general terms and conditions which have to be accepted to use most of the services online. It is anchored in our mind that it is enough to switch to incognito mode in case we need some extra privacy, not thinking about all the parties which can still legally access our browser data like the internet service provider or at work the employer.
\textbf{//look up how people really view privacy on the web}

But then again, from time to time, there are scandals like Facebooks data leak to Cambridge Analytica in March 2018. Gears start to grind and for just one second our consciousness was directed towards the big stream of information which flows by continuously in form of the internet. This big stream gives users the feeling of privacy, because who would be interested in one specific user? But scandals like the one Facebook caused disrupt this philosophy (or point of view). There are actually companies and people out there who are interested in this specific data. But blocking third party cookies should be enough to secure our privacy – right? Unfortunately, no. Because since short before 2010 there has been a new technique which is able to re-identify users and therefore collect data about users. 

This technique is called \textit{web browser fingerprinting} and can not be shout out or avoided. This is the reason which makes it so terrifying. Thus a user can not protect his browsing habits completely there are still methods which help to mitigate the effectiveness of this technique.
This bachelor thesis will discuss what exactly this tracking method is, how it works and explain what users can do to reduce its effectiveness. 
perceptions of people about their anonymity
(\textcite{mayer09}, p.17)

\newpage
\section{Objectives}
\subsection{Analysis of different web browser fingerprinting techniques}
One of the objectives of this bachelor thesis is to take a closer look as the different web browser fingerprinting techniques, which are currently in use. 
First some of the configurations and plugins which are used most are introduced to give a better explanation why the browser leaks information and what it is necessary for. Subsequently the different methods of fingerprinting are explained and based on them the different techniques.
Based on the foundation set in these subchapters the techniques will be shown based on an application scenario. \\ 
\\
The most important points which are discussed regarding this objective are:
\begin{itemize}
	\item used configurations and plug-ins
	\item fingerprinting methods
	\item fingerprinting techniques
	\item analysis of the techniques on the base of an application scenario
\end{itemize}

\subsection{Prototype}
Due to the outcome of the analysis of the different web browser fingerprinting techniques the best technique for the prototype will be chosen. The objective of the prototype is to show how web browser fingerprinting is implemented and how it works. Based on different configurations and plug-ins the prototype will create a hash which will help to re-identify users.

The prototype will be in form of a website which will include a fingerprinting script. It's functionality will cover following points:
\begin{itemize}
	\item reading configurations and plug-ins
	\item creating a hash
	\item re-identifying users
\end{itemize}

\section{Overview}
Following chapters discuss the stated content:\\\\
\hyperref[cha:foundation]{Chapter 2} discusses the basis of web browser fingerprinting, amongst others the different methods of fingerprinting. In order to compare them later on to analyse and eventually choose one of the methods to implement the prototype for chapter 4.\\
\hyperref[cha:ApplicationScenario]{Chapter 3} compares the fingerprinting techniques introduced in chapter 2 with the help of an application scenario. This comparison lays the basis for the decision which technique will be used for the prototype in chapter 5.\\
\hyperref[cha:requirements]{Chapter 4} will specify the requirements which are needed to outline a proper prototype.\\
\hyperref[cha:implementation]{Chapter 5} explains the implementation and design of the prototype.\\
\hyperref[cha:evaluation]{Chapter 6} concludes the previous chapter based on test cases and a proper evaluation.\\
\hyperref[cha:summary]{Chapter 7} sums up the bachelor thesis, summarizes the results and states possible prospects.




