\chapter{Requirements}
\label{cha:requirements}
as part of this work - develop a prototype
which technique is used - was decided in the previous section
this prototype is used to show how this particular technique is implemented
see in the sections concerned.
this section first discuses the use cases the prototype covers before it uses these to distinguish between functional and non functional requirements
as well as non-goals


\section{Use cases}
use cases are defined to distinguish between features which are needed and unnecessary over the top stuff

\begin{itemize}
	\item user can check if he is recognised, able to adapt, see how much the person is able to distinguish between others
	\item user is able to adapt and recheck if adaptions help to change his fingerprint - see which things need to be changed to change fingerprint
	see if unique at all
	\item server can check if recongnise users 
\end{itemize}

\section{Functional requirements}

\begin{itemize}
\item{measure data}
\item measure static data
\item run JS on client
\item filter important data or only acquire needed data
\item{create hash}
\item export data in form of .csv file
\item{re-identify users}
\item give feedback to user if he has been recongised
\item able to be used on different OS
\item feed back how distinguishable? or which things were used to distinguish the user
\end{itemize}

\section{Non functional requirements}

\begin{itemize}
\item good and pretty feedback for user
\end{itemize}

\section{Objective}
\begin{itemize}
	\item show how the technique is implemented
	\item show how it works
\end{itemize}

\section{Non Objective}
\begin{itemize}
	\item dont show heuristics and stuff
		\item adjust fingerprint with the help of heuristict to re-identify users after some changes
	\item stability - no errors
	\item user experience, good to use and able to config it
\end{itemize}
