\chapter{Requirements}
\label{cha:requirements}
As part of this bachelor thesis a prototype is composed  to show how fingerprinting techniques are implemented. This section first discusses the use case which is covered by the prototype before a distinction between functional and non functional requirements is made and non-goals are identified.

\section{Use case}
A use case is defined to distinguish between needed and obsolete features. The prototype is programmed to fulfill the following use case:

\begin{itemize}
	\item \textit{User recognition: } The prototype gives the user the chance to check if he has a unique fingerprint. On the users wish the website creates a fingerprint for the user. After the creation the user is informed if he has been recognised or if this was the first visit on the website. The user can try and change his fingerprint, then get back to the prototype and check if the website still recognises him.
\end{itemize}

\section{Functional requirements}

\begin{itemize}
\item \textit{Acquire data for browser fingerprinting: }
screen propterties, languages, timezone, plugins, et cetera.
\item \textit{Acquire data for canvas fingerprinting: }
pixel data of the rendered canvas.
\item \textit{Comparision of hashes: }Comparing already existing hashes from the local JSON file with the newly created hash.
\item \textit{Save hash: }Saving the created hash with the current date and time to a local JSON file.
\item \textit{Feedback for client: }Display a short message if the client has been recognised or not.
\item \textit{Compatibility: }The prototype should be able to work on multiple operating systems and in various browsers.
\end{itemize}

\section{Non functional requirements}

\begin{itemize}
\item \textit{Usability: }It is easy for the user to check his fingerprint and the feedback is understandable.
\item \textit{Correctness/Reliability: }The program creates the same fingerprint for a user in case nothing has changed.
\item \textit{Testability: }It is easy to test the prototype (creating a fingerprint and check back with the given time of the first visit).
\item \textit{Information: }The prototype gives a brief overview what the used techniques are about. 
\item \textit{Protability: }The prototype can easily be distributed to different devices.
\end{itemize}

\section{Objectives}
\begin{itemize}
	\item Use at least one fingerprinting technique to create the prototype.
	\item Test the prototype with multiple web browsers and operating systems.
\end{itemize}

\section{Non Objectives}
\begin{itemize}
	\item Use heuristics to check if a fingerprint has been changed.
	\item Inform the user which part of his hash is unique and which is identical to other users.
\end{itemize}
