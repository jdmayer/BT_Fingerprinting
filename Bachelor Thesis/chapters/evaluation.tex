\chapter{Testing and Evaluation}
\label{cha:evaluation}

This chapter depicts how the implemented fingerprinting prototype was tested and what outcome was achieved.

\section{Test cases}\label{sec:testCases}
Two testcases were tested on each fingerprinting technique to assure that the creation and comparison of the fingerprints work properly.

Those test cases are:
\begin{itemize}
	\item creation of a new fingerprint
	\item creation of a known fingerprint\\
\end{itemize}

The test procedure of both test cases is the same:
\begin{itemize}
	\item start server for prototype
	\item open prototype in browser
	\item make sure JavaScript is enabled
	\item click on button of wanted technique
	\item check response text below button\\
\end{itemize}
The key difference is what knowledge of the user the prototype already possesses. When testing with a new fingerprint, the prototype must recognise that the hash is not part of his fingerprint collection and save the newly created fingerprint. Further the correct response has to be displayed.\\\\
When testing with a known fingerprint, the prototype must recognise the fingerprint when comparing it to the collection. Therefore it may not add it again and display the correct user response.

\section{Tested devices}\label{sec:testdevices}

As stated in the requirements (\autoref{cha:requirements}), the test cases mentioned in \autoref{sec:testCases} have been tested on multiple browsers and operating systems. Both fingerprinting techniques (browser specific fingerprinting and canvas fingerprinting) have been tested on each device mentioned in the following subsections.

\subsection{Computers}
The test cases have been executed on 8 computers with multiple browsers. Each of the computers mentioned in \autoref{tab:PC} is a 64 bit system type and has executed the prototype in browsers mentioned in \autoref{tab:PCBrowser}.

\begin{table}[h]
	\centering
	\begin{tabular}{llll}
		Nr. & Operating System & Version & Graphics Processing Unit \\ \hline
		\rule{0pt}{15pt}1 & Windows 10 Home & 1803 & NVIDIA GTX 1050 \\
		2 & Windows 10 Home & 1803 & Intel(R) HD Graphics 520 \\
		3 & Windows 10 Enterprise & 1803 & NVIDIA GeForce GTX 960 \\
		4 & Windows 10 Enterprise & 1803 & Intel(R) UHD Graphics 630 \\
		5 & Windows 10 Pro & 1803 & NVIDIA GeForce MX 130 \\
		6 & Arch Linux & 5.0.2 & Mesa DRI Intel(R) HD Graphics 630 \\&&& (Kaby Lake GT2) \\
		7 & macOS Majave & 10.14.3 & Intel Iris Pro OpenGL Engine
	\end{tabular}
\caption{Tested computers}
\label{tab:PC}
\end{table}


\begin{table}[h]
	\centering
	\begin{tabular}{lll}
		Browser & Version & Tested devices \\ \hline
		\rule{0pt}{15pt}Microsoft Edge & 18.17763 & 1 \\
		Microsoft Edge & 17.17134 & 2, 3, 4, 5 \\
		Google Chrome & 73.0.3683.86 & 1, 4, 7 \\
		Google Chrome & 73.0.3683.75-2 & 6 \\
		Google Chrome & 72.0.3626.121 & 2, 5 \\
		Google Chrome & 63.0.3239.84 & 3 \\
		Mozilla Firefox & 66.0.2 & 1, 7 \\
		Mozilla Firefox & 65.0.2 & 2, 4 \\
		Mozilla Firefox & 60.5.2esr & 3 \\
		Safari & 12.0.3 & 7
	\end{tabular}
	\caption{Tested browsers with computers}
	\label{tab:PCBrowser}
\end{table}


\subsection{Mobile Phones}
Due to the easy distribution of the prototype the test could also have been executed on multiple mobile phones (see \autoref{tab:Phone}) and mobile browsers (see \autoref{tab:PhoneBrowser}).

\begin{table}[h]
	\centering
	\begin{tabular}{lllll}
		Nr. & Operating System & Graphics Processing Unit \\ \hline
		\rule{0pt}{15pt}1 & Android 9.0 & Adreno (TM) 630 \\
		2 & Android 9.0 & Mali-G72  \\
		3 & Android 7.0 & Adreno (TM) 506 \\
		4 & Android 7.0 & Mali-T830 \\
		5 & iOS 12.1.4 & Apple A12 GPU 
	\end{tabular}
\caption{Tested mobile phones}
\label{tab:Phone}
\end{table}


\begin{table}[h]
	\centering
	\begin{tabular}{lllll}
		Browser & Version & Tested devices \\ \hline
		\rule{0pt}{15pt}Google Chrome & 74.0.3729.157 & 4 \\
		Google Chrome & 73.0.3683.90 & 2, 3  \\
		Safari & 12.1.4 & 5 \\
		DuckDuckGo & 7.16.1 & 5 \\
		DuckDuckGo & 5.26.0 & 4 \\
		DuckDuckGo & 5.19.0 & 1 
	\end{tabular}
	\caption{Tested browsers with mobile phones}
	\label{tab:PhoneBrowser}
\end{table}

\section{Results}
\begin{itemize}
	\item \textit{Quantity: }Due to the execution of the test cases with the devices and browsers over 25 fingerprints could be acquired per technique.
	\item \textit{Correctness/Uniqueness: }Each of the acquired fingerprints was unique and could be used to re-identify the user.
	\item \textit{Compatibility: }The protoype can be executed successfully on each device and in each browser mentioned in \autoref{sec:testdevices}. 
	\item \textit{Portability:}The test cases can be executed on multiple devices, no matter if desktop computer, laptop or mobile phone.
	\item \textit{Weakness: }
	\begin{itemize}
		\item \textit{Zoom level: }Both fingerprints change in case the user created the fingerprint with a different zoom level, as the prototype does not take the zoom level into consideration. 
		\item \textit{Browser update: }In case of a browser version update the browser specific fingerprint changes, though the canvas fingerprint does not necessarily change.
		\item \textit{Cross browser: }As non of the techniques works across browsers the user can not be re-identified in case he visisted the prototype before in another browser.
		\item \textit{Plugins: }In case the user installs or de-installs a browser plugin the browser specific technique can not re-identify the user. In case the plug-in does not influence the graphic this does not concern the canvas fingerprint.\\
	\end{itemize}
\end{itemize}
Taken everything into consideration the prototype has a quote of 100\% in recognising known users with the same configurations. This has been tested with 13 different devices in 5 different browsers (with 16 versions, see \autoref{sec:testdevices}).\\\\
In case the users configuration have been changed since the fingerprint has been created, the user cannot be re-identified, as configuration alterations have not been taken into consideration for the prototype.


