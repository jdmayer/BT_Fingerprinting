\chapter{Foundation}
\label{cha:foundation}
This chapter will cover the basic knowledge about fingerprinting. First explaining the term fingerprint and what it can be used for. Further it will discuss the threat is poses as well as possible ways to reduce a fingerprint before covering the used configurations and plugins. After the base was layed the chapter closes with the descriptions of the different fingerprinting methods and techniques.

\section{Fingerprinting}
Web browser fingerprinting, also known as device fingerprinting, is the systematic collection of information to identify and later re-identify users. (\textcite{doty18}, p.3)(\textcite{amiunique})\\
This data tracking method works by obtaining data from the users web browser and using this data to create an unique fingerprint "hash". This hash is the key for later re-identifying the user. (\textcite{upi15}, p. 1)(\textcite{havens16}, p. 4)\\\\
With millions of computers and mobile devices in use and even more browsers, this identification method seems really unlikely to work at first sight. Nevertheless, it is an extremely reliable method for correlating data across websites or even web browsers with individual users. (\textcite{havens16}, p. 3)\\\\
Here is how it works:\\
There are currently over 7.5 billion people populating our planet. Imagine you have to identify a single person. By stating this specific person is male, you already cut the choice by half. Information about the ethnicity and age of this person as well as the time zone he lives in will narrow down the choice even more.\\
It's basically the same with web browser fingerprinting. The gender can be seen as an equivalent for the used operating system, the ethnicity as the browser and age as the browsers version. Information about the time zone and used language are easily acquirable by the browser and so are more specific details which help narrowing down the choice to nearly always one person.\\\\
This combination of properties can be obtained using code (time zone, screen resolution, plug-ins) and also be extracted from HTTP headers (user agent string). (\textcite{miele18}) \\\\
Web browser fingerprinting was first developed roughly before 2010 and due to its effectiveness quickly gained traction in the tracking industry.(\textcite{havens16}, p. 3) \\
Many of the fingerprinting methods can not be detected by the user and as it is extremely difficult for users to modify their web browsers in a way that they are less vulnerable to it (\textcite{amiunique})(\textcite{miele18}), makes it a dangerous tool for deanonymization attacks and maliciously-minded tracking programs. (\textcite{havens16}, p.3)\\
\begin{tcolorbox}
Example\\
Upon accessing a website, the fingerprinting script determines the users fingerprint and pushes it to a central data store. This data store is usually a shared data store which is used by multiple websites. So when the user accesses another website which also has access to this shared data store, the user will be recognized and the user profile can be updated. (\textcite{havens16}, p. 8)
\end{tcolorbox}

%\subsection{Entropy} - (\textcite{khan14})
\section{Utilisation}
\label{sec:utilisation}
In a study conducted in 2013 by Nikiforakis et al. they found that fingerprinting is a part of some of the most popular websites on the internet and that multiple hundred thousands of users are fingerprinted on a daily basis. (\textcite{nikiforakis13}, p.6) \\\\
The following paragraphs lists different ways in which fingerprinting can be utilized:\\\\
Constructive use
\begin{itemize}
	\item Security authentication\\
	Using web browser fingerprinting to correctly identify a device which is used to log into an account can be used to re-identify a user and help combat fraud or credential hijacking. (\textcite{upi15}, p.4)(\textcite{nikiforakis13}, p.2)
	\begin{tcolorbox}
	Example:\\
	Services can track the devices used to access an account and inform the user in case an unknown device tries to access it. 
	\end{tcolorbox}
\end{itemize}
Destructive use
\begin{itemize}
	\item Hacking\\
	Habits of users can be tracked without their knowledge and attackers can acquire specific knowledge about the user’s software setup. (\textcite{upi15}, p.4) 
	\begin{tcolorbox}
	Example:\\
	With the help of web browser fingerprinting a hacker can acquire knowledge about the users operating system and adjust his hacking method accordingly.
	\end{tcolorbox}
\end{itemize}
Con- and destructive use
\begin{itemize}
	\item Identifying criminals\\
	As a mean of tracking, fingerprinting can be used to track people or even criminals. But as seen in the previous examples this can used for and against users. 
	\begin{tcolorbox}
	Example: positive use\\
	When a user is harassed or stalked by another user the website can use web browser fingerprinting as a mean of blacklisting said user. 
	\end{tcolorbox}
	\begin{tcolorbox}
	Example: negative use\\
	Citizens of countries with a strict regime can be blacklisted for expressing criticism against the government.
	\end{tcolorbox}
	\item Commercial use\\
	Commercial Fingerprinting is used to track people’s habits and preferences. The acquired knowledge can be used to either help the user or direct him into a certain direction. 
	\begin{tcolorbox}
	Example: positive use\\
	A website picks up the user’s habit to look for climbing gear and suggests similar products.
	\end{tcolorbox}
	\begin{tcolorbox}
	Example: negative use\\
	A website registers that the user recently bought more expensive products and starts to only show items in this certain price range rather than more suitable but cheaper products.
	\end{tcolorbox}
\end{itemize}
The following paragraphs will summarize the findings of a study conducted by Nikiforakis et al. in cooperation with three commercial fingerprinting companies in order to test multiple websites for the use of fingerprinting scripts.\\\\
Nikiforakis et al. used the categorization of TrendMicro and McAfee on a list of 3804 domains. If a domain was neither analysed nor categorized by the both used services they were marked as untested and not used. Therefore only 59.2\% of the 3804 domains were included in the results of this testing. If one service declared a domain as unsafe and the other stated the opposite, it was accepted as safe. Some categories were given aliases and gathered together to a more generalized category.(\textcite{nikiforakis13}, pp.6)\\\\
\begin{figure}[H]
\centering
		\includegraphics[width=350pt, height=160pt]{Top10CategoriesNiki.png}
		\caption{Top 10 categories utilizing fingerprinting (\textcite{Top10Niki})}
		\label{Top10Niki}
\end{figure}
This figure shows 10 categories, 8 of which operate with user subscriptions. These sites usually are interested in preventing hijacking of user accounts and fraudulent activities.\\
The top two categories were identified as malicious (163 domains) or as spam (1063 domains). Nikiforakis et al. noticed that many domains belonging to one of these categories were parked websites which do not include any fingerprinting code. Further many "quiz" or "survey" websites were located which in a later step extract a user's personal details. The domains were categorized as malicious if they exploit vulnerable browsers or extract private data from users.\\\\
It was discovered that all these sites were found to include, as some point, fingerprinting code provided by one of the three commercial fingerprinting companies.\\
This observation, paired with the fact that all of these three studied providers stated that there must be set an appointment in order to acquire fingerprinting services, point to the possibility that fingerprinting companies work with dubious domains in order to expand their fingerprinting database.(\textcite{nikiforakis13}, p.7)


\section{Threat}
have a look at (\textcite{eckersley10})

As seen in the previous sub-chapter /ref{sec:utilisation} web browser fingerprinting does have its advantages as well as its disadvantages for the user. This section will have a closer look as the threat fingerprinting poses for the users privacy.
 
Web browser fingerprinting is a tracking method, which can correlate user's browsing activity within as well as across sessions without transparency or control. (\textcite{doty18}, p.3) Most fingerprinting techniques are obscure and users usually don't notice that they are being tracked.

The modern web exposes a number of details about the user’s device, either directly or discoverable through testing. This is largely a result of a greater need to comply with different device types, support for different features, or simply leaking data by including too much information in protocol requests.
(\textcite{havens16}, p. 4)
 
Privacy and security

\begin{itemize}
	\item Identify a user: concerns about surveillance, personal physical safety, concerns about discrimination against them based on what they read or write when using the web. (\textcite{doty18}, p.4)
	\item An online party can correlate multiple visits (on same or different sites) to develop a profile or history of the user. This may occur without the user’s knowledge or consent and tools such as clearing cookies do not prevent further correlation. Also allows tracking across origins – different sites may be able to combine information about a single user even where a cookie policy would block accessing of cookies between origins. (\textcite{doty18}, p.4)
	\item Tracking without transparency or user control: bf allows for collection of data about user activity without clear indications that such collection is happening. Allows for tracking of activity without clear or effective user controls. (typically cannot be cleared or re-set) /*like cookies*/ (\textcite{doty18}, p.4f)
\end{itemize}

\textbf{go to original source}
High risk: 
In 2010 Peter Eckersley, in association with the Electronic Frontier Foundation(EFF), conducted a study of browser fingerprinting techniques. –
**	Group of people, expected to have high awareness of privacy concerns. 
**	Out of 470,161 browsers – 93,6% instantaneously unique finerprint
**	Group -> privacy-conscious users, informed of the dangers of browser fingerprinting
**	Simple algorithm for detecting changes in fingerprints.
o	Check to see if there is any fingerprint which, aside from one non-matching field, is otherwise the same.
o	Not make guess if there were multiple potential matches
o	Made 65,56\% guesses – 99,1% right
(\textcite{havens16}, p.6f)

**	Deanonymization of Tor users
Tor enables JavaScript by default and if the user fails to turn it off, even Tor users can be tracked due to the information leaked by JavaScript. (\textcite{havens16}, p.9f)

\subsection{Under General Data Protection Regulation}
This threat posed by web browser fingerprinting as depicted above might be reduced by the \textit{General Data Protection Regulation (GDPR)}, which entered into force on May 25th 2018. This regulation imposed by the European Union (EU) intends to cover exactly this kind of hidden data collection which is used by web browser fingerprinting, by forcing companies to prove they have a legitimate reason for the utilization of any means of tracking.(\textcite{miele18})\\\\
Even though the GDPR avoids specifying technologies it provides general rules which should keep up with technological development. Due to this regulation browser characteristics are now to be treated like personal data. Personal data has a broad definition as any information that might be linked to an identifiable individual can be passed as such. Examples for such are not only the IP-Address and MAC-Address of users but also less specific features, including the combination of characteristics which web browser fingerprinting relies upon. (\textcite{miele18})\\\\
In order to be allowed to use fingerprinting legally the concerned entity has to complete the following steps:\\
\begin{itemize}
	\item show that the tracking does not violate "the fundamental rights and freedoms of the data subject, including privacy", 
	\item and is in line with "reasonable expectations of data subjects"
	\item further give a legitimate argument for its interest in tracking,
	\item and share details about the scope, purposes, and legal basis of the data processing with the person subjected to the fingerprinting\\
\end{itemize}
Due to this regulation, the only step the user has to take to avoid fingerprinting is to say "no".\\\\
Even though the rules imposed by this regulation seem to help prevent unwanted tracking, it only helps mitigate it. There will certainly be fewer entities making use of fingerprinting though web browser fingerprinting is not expected to disappear, no matter how high the penalties are on its illegal utilisation.\\
Anyway, the GDPR only applies on processed personal data of individuals living in the \textit{European Economic Area (EEA)} for commercial purposes, or for any purposes when the behaviour is within the EEA. There will always be companies which either think they can escape the consequences or which claim to have "legitimate interest" in tracking users. Further, no matter how strict regulations are, it always comes down to the user. Due to the plentiful requests for consent as they are found on the web nowadays many users are worn out and do not take a second look at the privacy policy regulated by the GDPR. (\textcite{miele18})

\section{Mitigation - max. 1,5 -2 pages}
As seen in the previous chapters there are many ways of utilizing web browser fingerprinting. Many of these possible applications pose a threat to the users privacy. So this brings up the question if there is a way to avoid this tracking method. Each of the researched articles about this topic stated the same. No. Unfortunately, there is no way to avoid being fingerprinted completely. There are only ways to try and reduce the risk of being tracked. (\textcite{web17})
The danger of web browser fingerprinting lays in its stateless nature. 
* hard to detect - if even possible
* impossible to opt-out
(\textcite{upi15}, p.4)
 \\\\
 \subsection{Best practice}
According to N. Doty (\textcite{doty18}, p.5), these are the best practices to mitigate the fingerprint we leave online: 
\begin{itemize}
	\item Avoid unnecessary or severe increases to fingerprinting surface
	\item Mark features that contribute to fingerprintability
	\item Specify orderings and non-functional differences
	\item Design APIs to access only the entropy necessary
	\item Enable graceful degradation for privacy-conscious users or implementers
	\item Avoid unnecessary new local state mechanisms
	\item Highlight any local state mechanisms to enable simultaneous clearing
	\item Limit permanent or persistent state\\
\end{itemize}
Further Doty mentions that this helps: 
\begin{itemize}
	\item Decreasing fingerprinting surface
	\item Increasing anonymity set
	\item detectable fingerprinting
	\item clearable local state
\end{itemize}
(\textcite{doty18}, p.7)\\

\subsection{Prevention?}
While N. Doty only listed ways to decrease the chance of being fingerprinted, R. Upathilake suggests the following steps to actually prevent the chance of being tracked via web browser fingerprinting (\textcite{upi15}, p.4):
\begin{itemize}
	\item Having all browser vendors agree on a single set of API calls to expose to the web applica-tions as well as internal implementation specifics;
	\item Having blocking tools which are maintained by doing regular web-crawls to detect tracking and incorporate blocking mechanisms into the tool;
	\item The introduction of a universal font list that a browser is limited to choose from for render-ing;
	\item Reporting unified and uncommunicative attributes
	\item Blocking or disabling js;
	\item Reducing the verbosity of the User-Agent string and the plug-in version;
	\item 	Having flash provide less system information and report only a standard set of fonts \\
\end{itemize}

\subsection{Canvas fingerprinting}
Is a difficult technique to automatically detect and prevent without false positives. A solution would be to turn to crowd-sourcing (sugg. By Acal et al.). A browser tool can de-fault to blocking all pixel data extraction attempts, but slowly over time, take into account user feedback and improve thetool to identify valid fingerprinting attempts. Other suggest-ed methods – having the browser add random pixel noise whenever pixels are extracted or having the browsers render scenes in a generic software renderer. These suggestions com e with a significant performance penalty and therefore, unacceptable for general use. The easiest method appears to require user approval if a script requests pixel data. Modern browsers already use this approach, for example with the html5 geolocation api. However, this introduces another permission dialogue which requires user interaction.
(\textcite{upi15}, p.4)

-- using private browsing modes or tor anonymity service
-- tor -> slower, less attrackive browsing experience
-- torbutton disables WebGL - still allows rendering to a <canvas> -> still partly vulnerable to canvas FP
-- mainstream users -> fingerprinting unavoidable consequence
(\textcite{mowery12}, p.1)


-- increasing the noise, but degrades the performance of <canvas> significantly for legitimate applications
-- same noise on multiple runs -> aid fingerprinting
-- therefore adding noise - not a feasible defence
-- one way would be: browser vendors - agree on a list of "canvas-safe" fonts
-- to support WebGL, ignore graphics card and render scenes in a generic software renderer
-- approach might be acceptable - performance impact not
-- easiest efficient defense: require user approval whenever a script requests pixel data 
-- modern browsers already implement this type for e.g. HTML5 geolocation APIs
(\textcite{mowery12}, p.10)

\subsection{Extensions and plug-ins}
The use of some common extensions and plug-ins were also mentioned in the liter-ature. The notable ones which seemed to aid in reducing fingerprintability were AdBlock Plus, Ghostery and NoScript. Nikiforakis et al. examined several highly rated extensions for firefox and chrome, which allowed changing the user-agent to appear as if sessions were from different computers. They noted in their study that these were inadequate and con-tained discrepancies which led to the browser being more distinguishable. These discrep-ancies allow fingerprinting scripts to increase the accuracy of the fingerprint since they are able to tell the presence of a specific extension. It should be noted that these results seem to be different from what was concluded by Yen et al. in their study. They suggested that changing the User-Agent string to one that corresponds to a populare browser version was a simple method to become less distinguishable. 
(\textcite{upi15}, p.4)

\subsection{The simplest solution}

-- browser extension -> automatically blocks active content e.g. JavaScript, Flash, or Silverlight applications (which don't deliver logically info to server)
-- these plugins - incl. NoScript (Firefox), ScriptBlock (Chrome) -> protection against canvas fingerprinting
-- using these plug-ins -> some web services or at least individual content -> stop working
-- not helpful if not sure if provider is trustworthy or not
-- these blockers can be directly used for fingerprinting the user
-- apart form script blocking solution
-- commonly-used browser and default settings
-- same with OS
-- high chance - no unique fingerprint and will be harder to track
(\textcite{web17})

We have also given some guidelines for developers to make their browsers resist
our fingerprinting algorithm. In particular, the verbosity of the UAS and the plugin
versions should be decreased, and a standard ‘substitute font set’ – preferably
supported by all browsers – should be introduced. Furthermore, users should be able
to set fake system properties in the browser in order to hide their real operating
system, time zone and screen resolution.
Finally, we have proposed some improvements for our method, and defined certain
directions for future experiments; an explicitly cross-browser experiment would
provide a high-precision dataset. 
(\textcite{boda11}, p.16)

\newpage
\section{Critical configurations and plug-ins}
-- max. 1/2 page per config

https://www.1and1.ca/digitalguide/online-marketing/web-analytics/browser-fingerprints-tracking-without-cookies/

\begin{comment}
\begin{figure}
	\centering
	\includegraphics[width=0.7\linewidth]{images/dangerousConfigs}
	\caption{}
	\label{fig:dangerousconfigs}
\end{figure}
(\textcite{mayer09}, p.43)
\end{comment}
As mentioned in the previous paragraph (chapter 2.1.) the scripts which use fingerprinting can read certain data via configuration settings or other observable characteristics from the web browser which enables them to create a specific fingerprint. (\textcite{doty18}, p.3)

The following paragraphs will introduce some of these certain information sources disclosed to third parties by the web browser and why they are used at all.


The information that browser fingerprinting reveals typically includes a mixture of HTTP headers (which are delivered as a normal part of every web request) and properties that can be learned about the browser using JavaScript code: your time zone, system fonts, screen resolution, which plugins you have installed, and what platform your browser is running on. Sites can even use techniques such as canvas or WebGL fingerprinting to gain insight into your hardware configuration.(\textcite{miele18})

\subsection{JavaScript}\label{JS}

JavaScript is enabled by default in all major browsers and by November 2018 JavaScript was included by 95\% of all websites online. (\textcite{w3techs18}).
-- therefore very useful tool - also because can query!
Js – the language which never works – for the developer

-- what is sputnik and test 262 ()

JS gives access to many browser-populated features like the plugins installed on the user’s device (\textcite{amiunique})

Identifying web browsers based on the underlying javascript engine
(\textcite{mulazzani13})

When JavaScript is disabled, websites won’t be able to detect the list of active plugins and fonts you use, and they also won’t be able to install certain cookies on your browser.
The disadvantage of disabling JavaScript is that websites won’t always function properly, be-cause it’s also used to make websites run smoothly on your device. This will impact your browsing experience.(\textcite{pixel18})

Browser fingerprinting, while a powerful tool, is often part of a larger suite of tracking techniques. Fingerprinting operates through the js engine as part of an analytics program. Any website that runs the fingerprinting analytics script will be able to track users through a shared data store.
(\textcite{havens16}, p.8)


Javascript has been standardized as ECMAScript [8],
and all major browsers implement it in order to allow
client-side scripting and dynamic websites. Traditionally,
Web developers use the UserAgent string or the navigator
object (i.e., navigator.UserAgent) to identify the client’s
Web browser, and load corresponding features or CSS files.
The UserAgent string is defined in RFC2616 [11] as a
sequence of product tokens and identifies the software as
well as significant subparts. Tokens are listed in order of their
significance by convention. The navigator object contains the
same string as the UserAgent string. However, both are by
no means security features, and can be set arbitrarily by the
user.
(\textcite{mulazzani13}, p.2)\\

SEE NIKIFORAKIS!!!!


\subsubsection{WebGL}\label{webgl}
%WebGL provides a JavaScript API for rendering 3D graphics in a <canvas> element. Modeled after OpenGL ES 2.0, WebGL is currently a draft specification and implemented and enabled in Chrome, Firefox, and Opera, as well as implemented but disabled in Safari. Each of these browsers provides a hardware-accelerated implementation, using the installed graphics hardware to render each frame. To mitigate serious misbehaviour and crashes, all of these browsers enable WebGL only for a whitelisted set of graphics cards and drivers.
%Current WebGL implementations expose their functionality through a separate canvas context (which will eventually be named “webgl”). The WebGL API is too complex to describe here in sufficient detail, but is stylistically similar to the desktop OpenGL API. It provides for vertex and fragment shaders, written in OpenGL Shading Language (GLSL), that, after compilation, run directly on the graphics card. WebGL also provides for OpenGL-style textures, as well as different lighting primitives. More advanced techniques, such as specular highlighting, bump mapping, and transparency, can be achieved through custom GLSL shaders. 2.4 SecurityImplications 
%(\textcite{mowery12}, p.3)


\subsection{Adobe Flash}\label{adobe}
Adobe Flash is a browser plug-in which provides different ways of delivering rich media content which could traditionally not be displayed using HTML. (\textcite{nikiforakis13}, p.3)


**	Criticized for poor performance, lack of stability
**	Newer technologies (HTML5) can potentially deliver what used to be only possible through flash – still available on the vast majority of desktops (\textcite{nikiforakis13}, p.3)
**	Returns all fonts with one simple call (\textcite{havens16}, p.5)
**	Its rich programming interface (API) provides access to many system-specific attributes (version of OS, list of fonts, screen resolution, timezone) (\textcite{amiunique})
**	Can be disabled without a negative impact on UEX – only impacts browsing experience on old websites(\textcite{pixel18}.)
**	Vulnerability, called “local storage objects” -> enables “zombie cookie” or “supercookie”. Almost impossible to remove. Populate numerous different storage areas with tracking in-formation. Even if cookies deleted from browser -> still come back. Other storage locations repopulate cookie as long as any existis – combined with brower fingerprinting, the cookies can track changes in fingerprints while also relying upon the fingerprint as a finale line of defense against the most privacy-conscious users. “cookie syncing” (\textcite{havens16}, p.8f)

SEE NIKIFORAKIS!!!!

\subsection{HTML 5 /Canvas}\label{html5}
-- HTML5 is the coding language used to build websites -> core fundamentals of every website
-- has the element "canvas"
-- originally html <canvas> element was used to draw graphics on a web page
(\textcite{pixel18})

Through the display of an HTML5 Canvas element, it is possible to collect small differences in the hardware on in the software configurations, thanks to slight differences in the image rendering between devices. The smallest pixel difference can be detected -> canvas fingerprinting 
(\textcite{amiunique})

%One of the most interesting new elements in HTML5, <canvas> provides an area of the screen which can be drawn upon programmatically. It enjoys widespread support, being available in the most recent versions of Chrome, Firefox, Internet Explorer, Opera, and Safari as well as Mobile Safari and Android Browser. The basic approach to drawing on a canvas is simple: acquire a graphics context, and use the context’s API to effect your changes. In the current HTML5 specification, the only defined context is“2d”. The 2d context provides basic drawing primitives such as fillRect, lineTo, and arc, as well as more complicated features such as B´ezier curves, color gradients, and copying in an existing image. 
%
%\textbf{refer to canvas fingerprinting}
%pixel extraction: In order for <canvas> to be a useful fingerprint, there must be some way to examine its behavior. Fortunately, <canvas> makes this extremely easy, providing several ways to inspect its data with pixel accuracy.
%(\textcite{mowery12}, p.2)
%
%First, the 2d context provides the method getImageData(). Given a rectangular region of the canvas, this method returns an ImageData object. Contained in this object are the RGBA values (as integers) for every pixel in the requested region. Second, the canvas object itself provides a toDataURL(type) method. When passed “image/png”, this method returns a data url consisting of the Base64 encoding of a PNG image containing the entire contents of the canvas. As this is a very convenient canvas-level method, we used this approach to extract data in our experiments. During black-box use of these fingerprints, the test suite could simply hash these data URLs, thereby removing the need to upload entire images from each client. It is worthwhile to note that these methods do preserve the same origin policy — if an image from a different origin has been drawn on this canvas, they will throw a SecurityError exception instead of returning pixel data. Therefore, our <canvas> fingerprints must only contain image resources that are under our control. 
%(\textcite{mowery12}, p.3)

\subsection{User agent}\label{UAS}
A user agent is a request header field in the hypertext transfer protocol (http) which is used for the communication between browser and website. (\textcite{xovi18}) It is automatically sent with each page call (\textcite{doty18}, p.6), except if it is specifically configured not to. (\textcite{xovi18})(\textcite{fielding14}, p.5.5.3)\\\\
The user agent contains the name and version of the used browser. In R. Fieldings work about http, this combination is also called product identifier. The product identifiers are listed in the order or their importance, whereas each of them can be followed by one or multiple comments. (\textcite{xovi18})(\textcite{fielding14}, p.5.5.3) This field value is often called user agent string. (\textcite{hoffman16})\\
\begin{tcolorbox}
Example 1: \\
Mozilla/5.0 (Windows NT 10.0; WOW64) AppleWebKit/537.36 (KHTML, like Gecko) Chrome/69.0.3497.102 Safari/537.36 Vivaldi/2.0.1309.37
\end{tcolorbox}
This string tells the receiving server, that the latest version (2.0.1309.37) of the Vivalid browser is used. In the comments (contained in the brackets) it gives away that the computer runs Windows 10 (Windows NT 10.0) on a 64-bit version (WOW64).\\
\begin{tcolorbox}
Example 2:\\
Mozilla/5.0 (Windows NT 10.0; Win64; x64) AppleWebKit/537.36 (KHTML, like Gecko) Chrome/64.0.3282.140 Safari/537.36 Edge/17.17134\\
\end{tcolorbox}
As well as with the other example, this browser (Edge with the version number 17.17134) also gives away the operating system and its version (same as before).\\\\
As shown in the examples above, not only the used browser name and its version are passed on but also information about the operating system and sometimes hardware. (\textcite{hoffman16}) (\textcite{xovi18}) The passed-on information is used to display different content in different browsers, on different oper-ating systems, or to gather statics (\textcite{hoffman16}) (\textcite{arntz17}) (\textcite{xovi18})\\\\
If wanted, changes on the user agent string are easily made. (\textcite{arntz17})
The problem with this configuration is, that the longer the user-agent field values become, the higher becomes the risk of being identified through for example fingerprinting. This is the case why the user should limit the addition to the user-agent string which can be requested by third parties. (\textcite{fielding14}, p.5.5.3)


\subsection{Others}
Quickly describe which else there are:
(\textcite{havens16}, p.5):
Language – Language the browser is in; used for localization
Screen Resolution – the with and height of the user’s screen
// here reference to why ratio is better -> zoom
Timezone – the local timezone of the user’s computer
DoNotTrack – weather the user has indicated they wish to opt-out of tracking
Installed Fonts – the fonts supported in the browser, typically retrieved from Adobe Flash but can also be detected by other means
Canvas Fingerprinting – The program attempts to render an image on the webpage using the HTML5 canvas. Subtle differences in the rendering of the image can lead to identification

(\textcite{havens16}, p.6)
Browser Plugins – a list of plugins installed in the browser
Cookies enabled – whether or not the user allows cookies on the page
Benchmark tests – evaluate how the browser’s js engine performs against a list of published bench marks. This information is often used to supplement that of the user agent string

//probably why it's needed, which can be blocked


\newpage
\section{Web browser fingerprinting methods}\label{sec:Methods}
\subsection{Active fingerprinting}
The active fingerprinting method needs to actively query information about the client that isn't automatically provided by the browser. (\textcite{web17}) To obtain additional characteristics the site needs to run a JavaScript code or use plug-ins like Adobe Flash which extend the browsers functionality. (\textcite{web17})((\textcite{doty18}, p.6)\\\\
Some of the additional characteristics which can be retrieved with this method are:
\begin{itemize}
	\item user's screen (width, height, resolution)
	\item window size
	\item enumerating fonts
	\item time zone
	\item plug-ins
	\item evaluating performance characteristics
	\item rendering graphical patterns\\
\end{itemize}
The only potential disadvantage active fingerprinting provides is that it can be detected on the client side. (\textcite{doty18}, p.6) Active fingerprinting can be discovered in case the user analyses the outgoing data packages, HTML or the JavaScript source code, which in the majority of cases does not happen. (\textcite{web17})\\\\
Active fingerprinting techniques
\begin{itemize}
	\item Canvas fingerprinting
	\item JavaScript Engine fingerprinting
	\item Cross-Browser fingerprinting
	\item \textbf{Browser specific fingerprinting?}
\end{itemize}

\subsection{Passive fingerprinting}
In contrast to active fingerprinting methods, passive fingerprinting does not need to execute any code on the client side. This method works based on observable characteristics which is contained in the header data of the IP packet by default. (\textcite{doty18}, p.6)(\textcite{web17}) There is no way for the user to learn if a website is using a passive fingerprinting method and secretly storing data.\\\\
Some of the provided characteristics are:
\begin{itemize}
	\item cookies
	\item http request headers
	\item ip address and port
	\item character sets (e.g. UTF-8)
	\item languages\\
\end{itemize}
??Passive fingerprinting techniques
\begin{itemize}
	\item \textbf{Browser specific fingerprinting?}
\end{itemize}


\section{Web browser fingerprinting techniques}
As described in \autoref{sec:Methods} there are different methods of fingerprinting. These methods are implemented in different techniques which will be discussed in this chapter.
\subsection{Browser specific fingerprinting}
%find AmIUnique paper --- about single browser fp
In 2010, P. Eckersley conducted a study in the project "Panopticlick" which checks how trackable users are via a from collected data generated fingerprint. (\textcite{upi15}, p.2) This study has shown that 94.2\% of the uses which use flash or java could be distinguished with browser fingerprinting. (\textcite{eckersley10}, p.2)\\
Possible even if the browser only reveals as much as its version and configuration information which is in most cases enough to create a distinct fingerprint.(\textcite{eckersley10}, p.16)

\subsubsection{How it works}
Browser specific fingerprinting is a technique which only uses browser-dependent features (like flash and JavaScript or be retrieved from a http request) to collect a number of common or and less-common known browser characteristics.\\

This retrieved browser information typically includes the installed fonts, plug-ins (including version numbers), user agent, http accept and screen resolution. (\textcite{upi15}, p.2)\\\\
As this kind of fingerprint pracitcally is just a concatination of this information it is easily affected by changes, like upgrades, newly installed features or extermal change of the system.
Usually, simple heuristics can be used to adjust the browser specific fingerprint accordingly.(\textcite{eckersley10}, p.4) (\textcite{upi15}, p.2)\\\\
The interesting fact about browsers is that sometimes, when they try to hide something, this helps even more to make the fingerprint distinct.

\begin{tcolorbox}
	Example: \\
	When a browser uses a flash blocking add-on, the browser still displays flash in the plugin list. When due to the blocking add-on the list of system fonts can't be obtained this creates a distinct fingerprint even though the flash was not explicitly detected. (\textcite{eckersley10}, p.4)
\end{tcolorbox}

Like many other web browser fingerprinting techniques, browser specific fingerprinting is not able to distinguish instances of identically configured devices. Further, other than cross browser fingerprinting, this is a single browser fingerprint method, which means it can not re-identify a user in another browser as the fingerprint may change across browsers and generally change easily(upgrade, plug-in, etc.). \textcite{upi15}, p.2)

\subsection{Canvas fingerprinting}
Canvas fingerprinting was first mentioned and implemented by Mowery and Shacham in 2012 as part of their work "Pixel Perfect: Fingerprinting Canvas in HTML5".\\
As the title teasers, this web browser fingerprinting technique works by using the new coding features in HTML5.(\textcite{pixel18}) Any website that runs JavaScript on the user's browser can use the HTML5<canvas> element and observe its rendering behaviour. There is no need for any special access to system resources. (\textcite{mowery12}, p.1)(\textcite{upi15}, p.2)\\\\
For rendering text fonts, webfonts were used as they are more distringuishable than expected. (\textcite{mowery12}, p.6) Further WebGL was used as  graphic cards leave a detectable fingerprint while rendering even the simplest scenes.(\textcite{mowery12}, p.8)\\
\begin{tcolorbox}
	Example text rendering: \\
	Out of 300 samples the text font Arial was rendered in 50 distinct ways. (\textcite{mowery12}, p.6)
\end{tcolorbox}
The outcome of their research and tests was that the canvas fingerprint is consistent but would change depending on hardware and software configuration. Further Mowery and Shacham discovered that the canvas fingerprint is unable to distinguish between users who use the exact same hadware and software.(\textcite{mowery12}, p.5)\\\\
Only 2 years after its first use a study conducted by Acar et al.(2014) ascertained that canvas fingerprinting is the most common form of fingerprinting with an approximately occurrence of 5.5\% in the top 100,000 websites. (\textcite{upi15}, p.2)

\subsubsection{How it works}
Canvas fingerprinting renders text and WebGL scenes onto an area of the screen using the HTML5<canvas> element programmatically and reads the pixel data back to generate a fingerprint. Through this process a 2D graphic context is obtained and a text or image is drawn to the canvas.(\textcite{upi15}, p.2)\\
The obtained canvas object provides a to-DataUrl(type) method which gives a data url consisting of a Base64 encoding of a png image which contains the canvas' contents. This Base 64 encoded pixel data url is used to create a hash.(\textcite{mowery12}, p.3)(\textcite{upi15}, p.2)\\\\
At least the operating system, browser version, graphics card, installed fonts, sub-pixel hinting and anti-aliasing all play a part in the final fingerprint.(\textcite{upi15}, p.2) \\\\
In their thesis K. Mowery and H. Shacham used two types of image comparisons: pixel-level difference and difference maps.(\textcite{mowery12}, pp.4)\\\\
Pixel-level difference works by setting each pixel's color to a channel-wise difference between tow images at that location. If the color is something else thatn transparent pure black it indicates that there is a difference between the two pixels. They set the alpha value of differing pixel to 255 to render it completly opaque. (\textcite{mowery12}, pp.4)\\\\
Difference maps works in a similar way, only that this method sets a pixel either black or white, depending on if they differ. If an image is purely white it indicates that they are the identical image. (\textcite{mowery12}, pp.4)

Canvas fingerprinting has problems by distinguishing identically configured devices, like any fingeprinting technique. Like all single browser fingerprinting method the users fingerprint changes over different browsers. \textcite{upi15}, p.2)

\begin{comment}
%It is transparent to the user. Our tests can be performed, offscreen, in a fraction of a second. There is no indication, visual or otherwise, that the user’s system is being fingerprinted.

Overall, our fingerprints do combine beneficially: 
among the 116 groups, our five extremely simple tests show a distribution entropy of 5.73 bits. We believe that more specialized and targeted tests could reveal even more, perhaps down to the exact installed graphics card, operating system, and browser family. (\textcite{mowery12}, p.8)

It is high-entropy. In 294 experiments on Amazon’s Mechanical Turk, we observed 116 unique fingerprint values, for a sample entropy of 5.73 bits. This is so even though the user population in our experiments exhibits little variation in browser and OS. (\textcite{mowery12}, p.1)

When writing their thesis Mowery et al. did not have access to various hundreds distrinct customer-level computer systems and therefore went to use Amazon Mechanical Turk and were able to collect samples from 300 distrinct member of this marketplace.(\textcite{mowery12}, pp.4) Running 5 fingerprinting tests in the background they collected metadata (browser's user agent string, WebGL-reported renderer, vendor, and version.
\end{comment}
\subsection{JavaScript Engine fingerprinting}

In 2011 P. Reischl et al. published their first paper about the use of the JavaScript Engine to uniquely identify a web browser and major version. Comparing it to the Panopticlick project (\textcite{eckersley10}) it was stated in the paper that the new approach is multiple times faster and error prone. (\textcite{reschl11}, p.2) Two years later, in 2013, the same team with some additional authors published another paper in which this fingerprint technique was described in more detail.(\textcite{mulazzani13}, p.1) \\\\
JavaScript Engine fingerprinting uses testsuits like Sputnik to determine the used browser and version. Compared to other methods this technique does not rely on the user agent string (which can be modified, see \autoref{UAS}) which makes it a more robost fingerprinting technique.(\textcite{mulazzani13}, p.1)

\subsubsection{How it works}
JavaScript Engine fingerprinting works by running test cases of conformance tests like Sputnik and test 262 to identify browsers and their major versions. Even though these test suits consist of multiple thousands test cases the browser only needs to fail one particular test which every other browser passed to be identified. Therefore these fingerprinting technique only needs a fraction of a second to be executed. (\textcite{mulazzani13}, p.3) \\\\
There are two methods which can be used to identify the webbrowser and the major version.\\\\
For a rather small test set the minimal fingerprint method can be used. First test262 is started for each browser in the set and the results are compared. For each browser a test is selected which only this specific browser failed. They select a test case which only one browser failed. This browser can be uniquely identified by this test and is therefore removed from the set. This process is redone for each browser until there is a unique fingerprint for each browser in the set or there is no test case which only one browser failed. If there is no unique fingerprint for a browser the selection is simply changed.(\textcite{mulazzani13}, p.3)\\\\
The figure below shows how the uniqueness of browsers and tests is obtained. The check marks mean, that the respective browser has passed the test. Each test which has a uniqueness of 1 can be used to uniquely identify a browser.
\begin{figure}[H]
	\centering
	\includegraphics[width=350pt, height=100pt]{MinimalFP.png}
	\caption{Example for the structure of a minimal fingerprint\\}
	\label{MinimalFP}
\end{figure}
If the test set is rather large it is more effective to use a binary decision tree. This method runs multiple test rounds to determine a web browser and major version. There is no need for a unique fingerprint as the tests split the browsers to identify them. Therefore there is no need to run a test for each browser and version like the  minimal fingerprint would which reduces the total number of executed tests. (\textcite{mulazzani13}, p.4)\\\\
The figure below \textbf{reference} give a good overview how such a tree can be structured. The leaf nodes pose as browsers, the inner nodes display the used tests and the edges indicate the success. (\textcite{mulazzani13}, p.4)
\begin{figure}[H]
	\centering
	\includegraphics[width=340pt, height=150pt]{DecisionTree.png}
	\caption{Example for the structure of a decision tree}
	\label{DecisionTree}
\end{figure}
%weaknesses?
\begin{comment}
--  having a comparable overhead for creating and collecting fingerprint samples.
-- implemented in just a few hundred lines of JS
-- undetectable for the user - CPU not stalled noticeably
(\textcite{mulazzani13}, p.2)
-- used to detect modified User-Agent strings, and can be used to reliably identify the browser of Tor Browser Bundle users.
(\textcite{upi15}, p.2)
\end{comment}
\subsection{Cross-Browser fingerprinting}

In 2011 Boda et al. conducted the first study about cross-browser fingerprinting, using a part of the IP address as part of the fingerprint. This is not optimal as the IP address can be modified by e.g. TOR or another proxy. (\textcite{boda11}, p.14)
Until Cao et al. proposed an improved fingerprint in 2017 it stayed the only cross-browser work.(\textcite{Cao17}, p.1)

\subsubsection{How it works}
Cross-Browser fingerprinting uses browser-independent features to be able to track a user regardless of what browser they are using. This can be done by relying only on the use of JavaScript and font detection. (\textcite{upi15}, pp.2)

The fingerprint is created from specific browser data with the help of JavaScript and server-side algorithms.(\textcite{boda11}, p.4)

To be able to track a user across different web browsers a set of browser-independent features is used as a basis of identification. Part of this set is for example (\textcite{boda11}, p.2)
\begin{itemize}
	\item networking information (ip address, hostname, TCP port number)
	\item application layer information (user agent string, name and version of the operating system, extensions)
	\item information gained by quering (list of fonts, screen resolution, timezone, plug-ins and their version)
\end{itemize}

The difference to other fingerprints is that the first two octets of the IP address are used, which usually remains constant even if the IP address of the client changes dynamically. The downside is that it may change when switching ISP or other service
(\textcite{boda11}, p.5)

-- use locality, short user id(script-generated identifier, derived from the first two octets of the IP address), usa, os, screen, timezone, basic fonts, all fonts, browser name and version
(\textcite{boda11}, p.5)

The downside of using the IP address as part of the fingerprint is that proxies modify it for privacy reasons. Boda et al. dealt with that problem saying that the other parameter suffice to identify a user.

Cao et al. improved this method by excluding the IP address, modifying the use of some data and adding some features.(\textcite{Cao17}, p.1) This technique forces the system to perform 36 tasks which return the required information within less than a minute.

This fingerprinting technique is based on different novel operating system and hardware level features (e.g. from graphic card, CPU, audio stack, and installed writing scripts). Many of which are exposed to JavaScript over web browser APIs, which are used to extracted them.(\textcite{Cao17}, pp.1)

99,24\%  successfully identify users 
83.24\% uniqueness with 91.44\% cross-browser stability(\textcite{Cao17}, p.2)
compared to Boda excluding ip address -68.98\% uniqueness with 84.64\% cross-browser stability.

Improvements towards Boda et al. technique. 
E.g. Change of the screen resolution. In browsers like Firefox and Internet Expolorer the resolution changes when scrolling. Therefore this method takes the zoom level into consideration and normalizes the width an dheight of the screen resolution.
E.g. DataUrl + JPEG are unstable across different browsers, therefore use a lossless format like PNG 
(\textcite{Cao17}, p.2)

Various rendering tasks (e.g. drawing curves and lines to client side) are sent. Also obtains operating system and hardware level information. The client side browser renders and returns the results (images, sound waves) which are converted into a hashes. In the meantime the browser collects browser-specific information.
FP composition - Fingerprint is generated from list of hashes which is intertwined with a mask through a "and" operation which creates the fingerprint hash.
The collected browser information corresponds to the mask. Each browser has its own mask. (\textcite{Cao17}, p.4)

Contrary to single browser techniques cross-browser fingerprinting has no problem identifying users across browsers. Anyway. It has the same weakness as all fingerprints: identically configured devices. (\textcite{upi15}, pp.2)
Weaknesses: (\textcite{upi15}, pp.2)

\begin{comment}
-- nr of CPU virtual cores - obtained by browser feature "hardwareConcurrency"
-- audiocontext - provides bundle of audio signal processing functionalities from signal generation to signal
--peak values and their corresponding frequencies are relatively stable
across browsers. 
-- create a list of bins with small steps on both the frequency and value axes, and map the peak
frequencies and values to the corresponding bins.
-- If one bin contains a frequency or value, we mark the bin as one and
otherwise zero: such list of bins serve as our cross-browser feature.
-- List of Fonts width and height of a certain string is measured to determine font type
-- Line, curve, and anti-aliasing. Line and curve are 2D features
supported by both Canvas (2D part) and WebGL.
--Vertex shader. A vertex shader, rendered by GPU and the
driver, converts each vertex in a 3D model to its coordinate in
a 2D clip-space.
-- Fragment shader. A fragment shader, rendered by GPU and
the driver as well, processes a fragment, such as a triangle
outputted by the rasterization, into a set of colors and a single
depth value.
--- MANY MORE - not write down all, als need to cut short - max. 1page per technique!!!!

%4 features for cross-browser fp (screen res., color depth, list of fonts, platform)
(\textcite{Cao17}, p.3)
\end{comment}