\chapter{Kurzfassung}

\begin{german}
Auf der ganzen Welt verstreut gibt es mehrere Milliarden Internetnutzer. Dem Großteil dieser Nutzer ist nicht bewusst, wie einfach es für Webseiten sein kann private Daten zu sammeln und weiterzuverkaufen. Skandale wie jener von Facebook in 2018 zeigen, wie die Nutzung einer Webseite Daten für Dritte zugänglich macht. Eine Art Informationen von Usern zu sammeln ist das sogenannte Web Browser Fingerprinting. Der breiten Masse gegenüber noch sehr unbekannt, stellt diese Tracking Methode jedoch einen einfachen Weg dar Benutzer auszuspionieren.	\\\\
Ein Teil der Lösung dieses Problems ist es, Internetnutzer auf die Gefahren des Trackings aufmerksam zumachen und ihnen Lösungswege zu bieten, um das Sammeln ihrer Daten zu verhindern. Hierfür müssen User eine ungefähre Ahnung haben, wie ihre Daten zu Dritten gelangen und einen Weg haben um zu prüfen, ob sie online trackbar sind. \\\\
Ziel dieser Arbeit ist es die Grundlagen für das Verständnis von Web Browser Fingerprinting zu bereiten. Weiters sollen Wege aufgezeigt werden, um die Wahrscheinlichkeit, eindeutig im Internet identifiziert zu werden, vermindert wird. Mithilfe eines Prototypen soll dargelegt werden, wie simpel es selbst für Amateure sein kann User eindeutig im Netz zu identifizieren. Internetnutzer sollen in der Lage sein mit dem Prototypen zu testen, ob sie einen eindeutigen Fingerprint haben, durch welchen sie getrackt werden können.
\end{german}